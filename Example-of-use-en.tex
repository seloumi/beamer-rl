\documentclass[arabic,english]{beamer-rl}
\usepackage{hologo} % pdflatex, xelatex, lualatex logos
\usepackage{verbatim}
\usepackage{listings} 
\usepackage{multicol}

% Themes without Navigation Bars
%\mode<presentation>{\usetheme{default}}
%\mode<presentation>{\usetheme{boxes}}
%\mode<presentation>{\usetheme{Bergen}}
%\mode<presentation>{\usetheme{Boadilla}}
\mode<presentation>{\usetheme{Madrid}}
%\mode<presentation>{\usetheme{AnnArbor}}
%\mode<presentation>{\usetheme{CambridgeUS}}
%\mode<presentation>{\usetheme{EastLansing}}
%\mode<presentation>{\usetheme{Pittsburgh}}
%\mode<presentation>{\usetheme{Rochester}}

% Themes with a Tree-Like Navigation Bar
%\mode<presentation>{\usetheme{Antibes}}
%\mode<presentation>{\usetheme{JuanLesPins}}
%\mode<presentation>{\usetheme{Montpellier}}

% Themes with a Table of Contents Sidebar
%\mode<presentation>{\usetheme{Berkeley}}
%\mode<presentation>{\usetheme{PaloAlto}}
%\mode<presentation>{\usetheme{Goettingen}}
%\mode<presentation>{\usetheme{Hannover}}

% Themes with a Mini Frame Navigation
%\mode<presentation>{\usetheme{Berlin}}
%\mode<presentation>{\usetheme{Ilmenau}}
%\mode<presentation>{\usetheme{Dresden}}
%\mode<presentation>{\usetheme{Darmstadt}}
%\mode<presentation>{\usetheme{Frankfurt}}
%\mode<presentation>{\usetheme{Singapore}}
%\mode<presentation>{\usetheme{Szeged}}

% Themes with Section and Subsection Tables
%\mode<presentation>{\usetheme{Copenhagen}}
%\mode<presentation>{\usetheme{Luebeck}}
%\mode<presentation>{\usetheme{Malmoe}}
%\mode<presentation>{\usetheme{Warsaw}}
%\usetheme{Cuerna}\usecolortheme{default}

\title{The beamer-rl class}

\author{Salim Bou}

\institute[]{%
{\color{blue} Repository: } https://github.com/seloumi/beamer-rl \par 
{\color{blue} Bug tracker: } https://github.com/seloumi/beamer-rl/issues
}

\date{{\selectlanguage{english} \today}}

\setbeamercovered{transparent=10}

% reduce space above and below verbatim env
\makeatletter
\preto{\@verbatim}{\topsep=0pt \partopsep=0pt }
\makeatother

\newenvironment{codeblock}{%
\selectlanguage{nil}
\smallskip\hfill%
\beamercolorbox[wd=\linewidth,sep=.3ex,dp=1ex]{block body}
\endgraf
\verbatim
}{%
\endverbatim
\endbeamercolorbox
\hfill
\smallskip
%\end{verbatim}
}


\begin{document}

\selectlanguage{english}

\parskip=6pt

\begin{frame}
\titlepage
\end{frame}

\begin{frame}
\frametitle{\selectlanguage{english}\contentsname}
\tableofcontents
\end{frame}


\section{Introduction}

\begin{frame}[fragile]
\frametitle{Introduction}
Creating beamer presentation for languages with script from right to left   (like arabic) using \hologo{pdfLaTeX} or \hologo{XeLaTeX} still poses many problems due to bugs not currently resolved especially for colors.

The \hologo{LuaTeX} team set solutions for  these issues thanks to them and to \textit{Javier~Bezos} for  his works on the package \verb|babel| and \verb|bidi| writing    

This class provides patchs of some beamer templates and commands to create  
 right to left beamer presentation, the class call babel with \verb|bidi=basic| option and require \hologo{LuaLaTeX} engine

\end{frame}

\section{How to use beamer-rl}

\begin{frame}[fragile,allowframebreaks]
\frametitle{How to use beamer-rl}

\begin{codeblock}
\documentclass{beamer-rl}

% import language
\babelprovide[import=ar-DZ, main]{arabic} 

\usetheme{Madrid}

\begin{document}
...
\end{document}
\end{codeblock}

We get a similar result by adding the main language of the presentation (language with right-to-left script)
  as option of class as follows:

\begin{codeblock}
\documentclass[arabic]{beamer-rl}

\usetheme{CambridgeUS}

\begin{document}
...
\end{document}
\end{codeblock}

We can also add more language options that the command \verb|\babelprovide|  provides as follows:

\begin{codeblock}
\documentclass[arabic={mapdigits}]{beamer-rl}

% equivalent to
% \babelprovide[import,main,mapdigits]{arabic}

\end{codeblock}

The class  define in the same way as  options (languages supported by the package \verb|babel| with script from right to left) 

\begin{multicols}{3}
\verb|arabic|\\
\verb|arabic-dz|\\
\verb|arabic-tn|\\
\verb|arabic-ma|\\
\verb|arabic-eg|\\
\verb|arabic-sa|\\
\verb|arabic-iq|\\
\verb|arabic-sy|\\
\verb|arabic-lb|\\
\verb|arabic-ps|\\
\verb|arabic-jo|\\
\verb|centralkurdish|\\
\verb|hebrew|\\
\verb|kashmiri|\\
\verb|mazanderani|\\
\verb|malayalam|\\
\verb|northernkurdish-arab|\\
\verb|pashto|\\
\verb|persian|\\
\verb|punjabi-arab|\\
\verb|syriac|\\
\verb|urdu|\\
\verb|uyghur|\\
\verb|uzbek-arab|\\
\verb|yiddish|\\
\end{multicols}


\end{frame}

\section{Some notes}

\begin{frame}[fragile,allowframebreaks]
\frametitle{Some notes}

\begin{itemize}

\item
The class define \verb:Amiri: as default sans serif font, we can 
modify this in the preambule  with   

\begin{codeblock}
\babelfont{sf}{<font name>}
\end{codeblock}

\item
The class defines  option \verb|layout| which passes its content to \verb|babel| 

\medskip 
 
\begin{codeblock}
\documentclass[layout={<babel layout>}]{beamer-rl}
\end{codeblock}

More on the subject can be found in the manual of  \verb|babel| package 
\href{http://mirrors.ctan.org/macros/latex/required/babel/base/babel.pdf}{\beamergotobutton{link}}

\bigskip

\item
In some cases you  need to use \verb:\babelsublr: command from \verb:bebel: 
package to insert a left to right text within your right to left text, e.g if you need to insert a \verb:pspicture: drawing in RTL context 

\begin{codeblock}
\bebelsublr{LTR context ... }
\end{codeblock}

\end{itemize}

\end{frame}

\section{pgfpages-rl package}

\begin{frame}[fragile]
\frametitle{pgfpages-rl package}

\verb:pgfpages-rl: adds to \verb:pgfpages: the ability to support TRT pagedir, the package requires \hologo{LuaLaTeX} engine. 
It can also be used with other document classes besides \verb:beamer-rl: 

\begin{codeblock}
\documentclass{beamer-rl}
\babelprovide[import=ar-DZ, main]{arabic}
\usetheme{Warsaw}
\usepackage{pgfpages-rl} % adapt pgfpages to TRT pagedir
\setbeamertemplate{note page}[]
\setbeameroption{show notes on second screen=right}
\begin{document}
...
\end{document}

\end{codeblock}


\end{frame}

\section{Examples}

\begin{frame}[plain]{}

\fontsize{70}{60}\selectfont\centerline{Examples}

\end{frame}

\subsection{Blocks}

\begin{frame}[fragile]
\frametitle{Blocks}

\begin{codeblock}
\setbeamertemplate{blocks}[default]
\end{codeblock}

\setbeamertemplate{blocks}[default]

\begin{block}{Lorem}
  \selectlanguage{nil} 
  On 21 April 1820, during a lecture, Ørsted
  noticed a compass needle deflected from magnetic north when an
  electric current from a battery was switched on and off.
\end{block}

\begin{codeblock}
\setbeamertemplate{blocks}[rounded][shadow=true]
\end{codeblock}

\setbeamertemplate{blocks}[rounded][shadow=true]

\begin{block}{Lorem}
  \selectlanguage{nil} 
  On 21 April 1820, during a lecture, Ørsted
  noticed a compass needle deflected from magnetic north when an
  electric current from a battery was switched on and off.
\end{block}

\end{frame}

\subsection{Lists}

\begin{frame}[fragile,allowframebreaks]
\frametitle{enumerate, itemize}

\selectlanguage{arabic}
\begin{columns}[t,onlytextwidth]

\begin{column}{2cm}
\begin{enumerate}
\item First 
\item Second 
\end{enumerate}
\end{column}
\begin{column}{8.8cm}
\begin{codeblock}
\setbeamertemplate{enumerate item}[ball]
\begin{enumerate}
\item First 
\item Second 
\end{enumerate}
\end{codeblock}
\end{column}

\end{columns}

\vfill
\setbeamertemplate{itemize item}[triangle]

\begin{columns}[t,onlytextwidth]

\begin{column}{2cm}
\begin{itemize}
\item First
\item Second 
\end{itemize}
\end{column}
\begin{column}{8.8cm}
\begin{codeblock}
% in RTL context
\setbeamertemplate{itemize item}[triangle]
\begin{itemize}
\item First
\item Second 
\end{itemize}
\end{codeblock}
\end{column}

\end{columns}

\framebreak

\selectlanguage{nil}

\begin{columns}[t]
\begin{column}{2cm}
\begin{itemize}
\item First
\item Second 
\end{itemize}
\end{column}
\begin{column}{8.5cm}
\begin{codeblock}
% in LTR context 
\setbeamertemplate{itemize item}[triangle]
\begin{itemize}
\item First
\item Second 
\end{itemize}
\end{codeblock}
\end{column}
\end{columns}

\end{frame}

\subsection{Hyperlinks}

\begin{frame}[fragile]
\frametitle{Hyperlinks}

\selectlanguage{arabic}

\begin{itemize}\shapemode2
\item<1-> First.
\item<2-> Second.
\end{itemize}

\hyperlink{jumptofirst}{\beamergotobutton{return to first slide}}
\hypertarget<1>{jumptofirst}{}

\pause

\begin{codeblock}
\hyperlink{jumptofirst}
{\beamergotobutton{return to first slide}}
\hypertarget<1>{jumptofirst}{}
\end{codeblock}

\end{frame}


\subsection{Theorems}

\begin{frame}
\frametitle{Theorems}

\framesubtitle{The proof uses \textit{reductio ad absurdum}.}

\selectlanguage{arabic}

\begin{theorem}
There is no largest prime number.
\end{theorem}
\begin{proof}
\begin{enumerate}[<+-| alert@+>]
\item Suppose $p$ were the largest prime number.
\item Let $q$ be the product of the first $p$ numbers.
\item Then $q+1$ is not divisible by any of them.
\item But $q + 1$ is greater than $1$, thus divisible by some prime
number not in the first $p$ numbers.\qedhere
\end{enumerate}
\end{proof}

\end{frame}

\subsection{Zooming}

\selectlanguage{arabic}

\setbeamertemplate{frametitle}[default][right]

\begin{frame}[fragile]
\frametitle{Zooming}


\framezoom<1><2>[border=2](1cm,1cm)(2cm,2cm)
\pgfimage[height=4cm]{example-image}

\begin{codeblock}
\framezoom<1><2>[border=2](1cm,1cm)(2cm,2cm)
% (1cm,1cm)=(<upper right x>,<upper right y>)
% (2cm,2cm)=(<zoom area width>,<zoom area depth>)
\pgfimage[height=5cm]{example-image}
\end{codeblock}

\end{frame}

\end{document}
