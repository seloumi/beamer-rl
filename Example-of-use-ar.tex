\documentclass[babel={layout=lists}]{beamer-rl}
\usepackage{hologo} % pdflatex, xelatex, lualatex logos 

\babelprovide[import=ar-DZ, main]{arabic}
\babelfont{sf}{Amiri}

% Themes without Navigation Bars
%\mode<presentation>{\usetheme{default}}
%\mode<presentation>{\usetheme{boxes}}
%\mode<presentation>{\usetheme{Bergen}}
%\mode<presentation>{\usetheme{Boadilla}}
%\mode<presentation>{\usetheme{Madrid}}
%\mode<presentation>{\usetheme{AnnArbor}}
\mode<presentation>{\usetheme{CambridgeUS}}
%\mode<presentation>{\usetheme{EastLansing}}
%\mode<presentation>{\usetheme{Pittsburgh}}
%\mode<presentation>{\usetheme{Rochester}}

% Themes with a Tree-Like Navigation Bar
%\mode<presentation>{\usetheme{Antibes}}
%\mode<presentation>{\usetheme{JuanLesPins}}
%\mode<presentation>{\usetheme{Montpellier}}

% Themes with a Table of Contents Sidebar
%\mode<presentation>{\usetheme{Berkeley}}
%\mode<presentation>{\usetheme{PaloAlto}}
%\mode<presentation>{\usetheme{Goettingen}}
%\mode<presentation>{\usetheme{Hannover}}

% Themes with a Mini Frame Navigation
%\mode<presentation>{\usetheme{Berlin}}
%\mode<presentation>{\usetheme{Ilmenau}}
%\mode<presentation>{\usetheme{Dresden}}
%\mode<presentation>{\usetheme{Darmstadt}}
%\mode<presentation>{\usetheme{Frankfurt}}
%\mode<presentation>{\usetheme{Singapore}}
%\mode<presentation>{\usetheme{Szeged}}

% Themes with Section and Subsection Tables
%\mode<presentation>{\usetheme{Copenhagen}}
%\mode<presentation>{\usetheme{Luebeck}}
%\mode<presentation>{\usetheme{Malmoe}}
%\mode<presentation>{\usetheme{Warsaw}}

\usecolortheme{spruce}

\title{{\ttfamily beamer-rl} class}

\author{Salim Bou}

\institute[]{%
{\color{blue} Repository: } https://github.com/seloumi/beamer-rl \par 
{\color{blue} Bug tracker: } https://github.com/seloumi/beamer-rl/issues
}

\date{\today}

\setbeamercovered{transparent=10}
\newtheorem*{thm}{نظرية.}

\def\cs#1{\babelsublr{\texttt{\textbackslash#1}}}

\begin{document}

\parskip=6pt

\begin{frame}
\titlepage
\end{frame}

\begin{frame}
\frametitle{\contentsname}
\tableofcontents
\end{frame}

\section{مدخل}

\begin{frame}[fragile]
\frametitle{مدخل}
انشاء عرض بيمر عربي (اتجاه النص من اليمين لليسار) اعتمادا على 
 \hologo{pdfLaTeX} أو \hologo{XeLaTeX} مازال يعترضه الكثير من المشاكل والمعوقات خاصة ما يتعلق بالألوان والروابط والتي لم يوجد لها حلولا~بعد. 

فريق  \hologo{LuaTeX} 
أوجد حلولا لهذه المشاكل، الشكر لهم ولـ 
\textit{Javier~Bezos}
لأعماله بالحزمة
\verb|babel| وخصوصا الكتابة بالاتجاهين (\verb|bidi| writing)    

هذه الفئة (class) تعدل في بعض اوامر وتعليمات beamer لغرض انشاء عروض من اليمين إلى اليسار (العربية على سبيل المثال)، الفئة تستدعي \verb|babel| مع الخيار 
\verb|bidi=basic-r|
 والمعالجة تتطلب استخدام 
\hologo{LuaLaTeX} 

\end{frame}

\section{كيفية استعمال الفئة}

\begin{frame}[fragile]
\frametitle{كيفية استعمال الفئة}

\selectlanguage{nil}

\begin{verbatim}
\documentclass{beamer-rl}
\babelprovide[import=ar-DZ, main]{arabic}
\babelfont{sf}{Amiri}

\mode<presentation>{\usetheme{Warsaw}}
\begin{document}
...
\end{document}

\end{verbatim}

\end{frame}

\section{أمثلة}
\subsection{الإطارات}

\begin{frame}[fragile]
\frametitle{الإطارات}

{\selectlanguage{nil}
\verb:\setbeamertemplate{blocks}[default]:
}

\setbeamertemplate{blocks}[default]


\begin{block}{أورستد}
  لاحظ هانز أورستد في 21 أبريل 1820 وهو يُعد أحد التجارب أن إبرة
  البوصلة تنحرف عن اتجاهها نحو الشمال عندما كان يغلق ويفتح التيار في
  دائرة كهربائية يُعدها.
\end{block}

{\selectlanguage{nil}
\verb:\setbeamertemplate{blocks}[rounded][shadow=true]:
}

\setbeamertemplate{blocks}[rounded][shadow=true]

\begin{block}{أورستد}
  لاحظ هانز أورستد في 21 أبريل 1820 وهو يُعد أحد التجارب أن إبرة
  البوصلة تنحرف عن اتجاهها نحو الشمال عندما كان يغلق ويفتح التيار في
  دائرة كهربائية يُعدها.
\end{block}

\end{frame}

\subsection{القوائم}

\begin{frame}[fragile]
\frametitle{enumerate, itemize}

\begin{enumerate}
\item فيزياء تطبيقية
\item فيزياء تجريبية
\item فيزياء نظرية
\end{enumerate}

\setbeamertemplate{itemize item}[triangle]

{\selectlanguage{nil}
\verb|\setbeamertemplate{itemize item}[triangle]|
}

\begin{itemize}
\item فيزياء تطبيقية
\item فيزياء تجريبية
\item فيزياء نظرية
\end{itemize}

\selectlanguage{nil}


\begin{itemize}
\item first item
\item second item
\item third item
\end{itemize}

\end{frame}

\subsection{الروابط}

\begin{frame}
\frametitle{الروابط}
\begin{itemize}
\item<1-> العنصر الأول.
\item<2-> العنصر الثاني.
\item<3-> العنصر الثالث.
\end{itemize}
\hyperlink{jumptosecond}{\beamerreturnbutton{الرجوع إلى الشريحة الثانية}}
\hypertarget<2>{jumptosecond}{}

\end{frame}


\subsection{النظريات}

\begin{frame}
\frametitle{النظريات}

\framesubtitle{The proof uses \textit{reductio ad absurdum}.}
\begin{thm}
There is no largest prime number.
\end{thm}
\begin{proof}
\begin{enumerate}[<+-| alert@+>]
\item Suppose $p$ were the largest prime number.
\item Let $q$ be the product of the first $p$ numbers.
\item Then $q+1$ is not divisible by any of them.
\item But $q + 1$ is greater than $1$, thus divisible by some prime
number not in the first $p$ numbers.\qedhere
\end{enumerate}
\end{proof}

\end{frame}

\subsection{التكبير}

\begin{frame}[fragile]
\frametitle{التكبير}

\framezoom<1><2>[border=2](2cm,2cm)(2cm,2cm)
\pgfimage[height=5cm]{example-image}

\selectlanguage{nil}

\begin{verbatim}
\framezoom<1><2>[border=2](2cm,2cm)(2cm,2cm)
\pgfimage[height=5cm]{example-image}
\end{verbatim}
\end{frame}

\section{بعض الملاحظات}

\begin{frame}[fragile]
\frametitle{بعض الملاحظات}

\begin{itemize}
\item
يمكن اضافة كل الخيارات التي تتيحها الفئة \verb:beamer: عند استدعاء الفئة \verb:beamer-rl:

كما يمكن تمرير خيارات اضافية للحزمة 
 \verb:babel: 
 عند استدعاء الفئة \verb:beamer-rl: على الشكل:

\medskip 
 
{\selectlanguage{nil}
\verb:\documentclass[babel={<babel options>}]{beamer-rl}:
}

\medskip

\item
الفئة 
 \verb:beamer-rl: تقوم بتبادل لكل من التعليمتين  \cs{blacktriangleright} و   \cs{blacktriangleleft} في حالة نص من اليمين لليسار

\bigskip

{\selectlanguage{nil}
\centering
\begin{tabular}{c|cc}
\hline
 &  \verb:\blacktriangleright: & \verb:\blacktriangleleft:   \\
\hline 
LTR context & \blacktriangleright & \blacktriangleleft \\
\hline
RTL context & {\selectlanguage{arabic}\blacktriangleright} & {\selectlanguage{arabic}\blacktriangleleft} \\
\hline
\end{tabular}
\par
}

\bigskip
 
 
\item
في بعض الحالات يمكن استعمال التعليمة 
 \cs{babelsublr} التي توفرها الحزمة  \verb:bebel: 
لادراج نص من اليسار لليمين (لاتيني) في وسط نص من اليمين لليسار،
 على سبيل المثال  في حال اردنا ادراج رسم  
 \verb:pspicture: ضمن نص من اليمين لليسار. 
\end{itemize}


\end{frame}

\end{document}
